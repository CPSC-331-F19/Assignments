\documentclass[12pt]{article}
 
\usepackage[margin=1in]{geometry} 
\usepackage{amsmath,amsthm,amssymb,scrextend}
\usepackage{mathtools}
\usepackage{fancyhdr}
\usepackage{enumitem}
\usepackage{fancyvrb}
\pagestyle{fancy}
 
\newcommand{\N}{\mathbb{N}}
\newcommand{\Z}{\mathbb{Z}}
\newcommand{\I}{\mathbb{I}}
\newcommand{\R}{\mathbb{R}}
\newcommand{\Q}{\mathbb{Q}}
\newcommand{\lbr}{\lbrace}
\newcommand{\rbr}{\rbrace}
\DeclarePairedDelimiter\ceil{\lceil}{\rceil}
\DeclarePairedDelimiter\floor{\lfloor}{\rfloor}
\renewcommand{\qed}{\hfill$\blacksquare$}
\let\newproof\proof
 
\newenvironment{theorem}[2][Theorem]{\begin{trivlist}
\item[\hskip \labelsep {\bfseries #1}\hskip \labelsep {\bfseries #2.}]}{\end{trivlist}}
\newenvironment{lemma}[2][Lemma]{\begin{trivlist}
\item[\hskip \labelsep {\bfseries #1}\hskip \labelsep {\bfseries #2.}]}{\end{trivlist}}
\newenvironment{problem}[2][Problem]{\begin{trivlist}
\item[\hskip \labelsep {\bfseries #1}\hskip \labelsep {\bfseries #2.}]}{\end{trivlist}}
\newenvironment{exercise}[2][Exercise]{\begin{trivlist}
\item[\hskip \labelsep {\bfseries #1}\hskip \labelsep {\bfseries #2.}]}{\end{trivlist}}
\newenvironment{reflection}[2][Reflection]{\begin{trivlist}
\item[\hskip \labelsep {\bfseries #1}\hskip \labelsep {\bfseries #2.}]}{\end{trivlist}}
\newenvironment{proposition}[2][Proposition]{\begin{trivlist}
\item[\hskip \labelsep {\bfseries #1}\hskip \labelsep {\bfseries #2.}]}{\end{trivlist}}
\newenvironment{corollary}[2][Corollary]{\begin{trivlist}
\item[\hskip \labelsep {\bfseries #1}\hskip \labelsep {\bfseries #2.}]}{\end{trivlist}}
\newenvironment{solution}{\renewcommand\qedsymbol{$\blacksquare$}\begin{proof}[Solution]}{\end{proof}}

\usepackage{minted}

\begin{document}

% --------------------------------------------------------------
%                         Start here
% --------------------------------------------------------------

\title{Assignment 3 \\ CPSC 331} % Title and subtitle
\author{Guransh Mangat, 30061719 \\ Daniel Contreras, 10080311 \\ Steven Ferguson, 30037518\\}
\date{}


\lhead{CPSC 331}
\chead{Assignment 3}
\rhead{November 2019}

\maketitle{}

\newpage

% --------------------------------------------------------------
%                         1
% --------------------------------------------------------------
\begin{problem}{1}
\end{problem}

\begin{solution}

\textbf{Claim:} The hybrid sorting algorithm correctly solves the sorting problem.
$\newline$
\textbf{Method of Proof:} Induction on length \texttt{n = A.length} of the input ArrayList - using strong form of mathematical induction. Consider \texttt{A.length = THRESHOLD} for the base case. 
\begin{enumerate}
    \item By examining the algorithm, the test at line 1 passes and the execution continues with line $2$ and $3$. Eventually executing line $4$ and halting while returning sorted ArrayList B as output. This is sufficient to complete the basis. 
    
    \item An application of the inductive hypothesis when the steps at line $8$ and $9$ are executed and the correctness of \texttt{merge} algorithm at line $10$ is sufficient to complete the inductive step. 
\end{enumerate}
\end{solution}

%------------------------------------------------------------------------
%                         Problem 2
%------------------------------------------------------------------------
\begin{problem}{2}
\end{problem}

\begin{solution}
\end{solution}

%------------------------------------------------------------------------
%                         Problem 3
%------------------------------------------------------------------------
\begin{problem}{3}

Proof by induction on $n$: Let \texttt{THRESHOLD=2}, then it is necessary to consider the following base cases on \texttt{THRESHOLD}: \newline

\texttt{THRESHOLD=0}: $T_{THRESHOLD}(0) = a0^2 + \beta0 + \delta \leq a0^2 + \beta0 + \delta = T(0, 2)$

\texttt{THRESHOLD=1}: $T_{THRESHOLD}(1) = a1^2 + \beta1 + \delta \leq a1^2 + \beta1 + \delta = T(1, 2)$

\texttt{THRESHOLD=2}: $T_{THRESHOLD}(2) = a2^2 + \beta2 + \delta \leq a2^2 + \beta2 + \delta = T(2, 2)$ \newline


\textbf{Inductive hypothesis:} Let $k$ be the length of the array and suppose $T_{\texttt{THRESHOLD}}(i) < T(i, \texttt{THRESHOLD})$ for all $0 \leq i \leq \texttt{THRESHOLD}$. \newline

\textbf{Inductive Claim:} Show that $T_{\texttt{THRESHOLD}}(k+1) \leq T((k+1), \texttt{THRESHOLD})$ for all $k > 2$. \newline

We have:

\begin{flalign*}
    T_{\texttt{THRESHOLD}}(k+1) & = T_{\texttt{THRESHOLD}}(\ceil{\frac{k+1}{2}}) + T_{\texttt{THRESHOLD}}(\floor{\frac{k+1}{2}}) + \gamma(k+1) + \zeta \\
    & \leq T(\ceil{\frac{k+1}{2}}, \texttt{THRESHOLD}) + T(\floor{\frac{k+1}{2}}, \texttt{THRESHOLD}) + \gamma(k+1) + \zeta &\text{(by I.H.)} \\
   & = T((k+1), \texttt{THRESHOLD}) &\text{as required.} \\
\end{flalign*}

\end{problem}

%------------------------------------------------------------------------
%                         Problem 4
%------------------------------------------------------------------------
\begin{problem}{4}
\end{problem}

\begin{solution}
\end{solution}

%------------------------------------------------------------------------
%                         Problem 5
%------------------------------------------------------------------------
\begin{problem}{5}
\end{problem}

\begin{solution}
The implementation of the ``Merge Sort" algorithm has more importance on the runtime of our hybrid sorting algorithm. We can replace the ``Merge Sort" algorithm with ``Quick Sort" algorithm in order to improve the performance. Despite ``Quick Sort" having a $O(n^2)$ in the worst case, this can easily be avoided by choosing an optimal pivot. In addition, quick sort requires less space requirements than merge sort. 
\end{solution}

%------------------------------------------------------------------------
%                         End
%------------------------------------------------------------------------

\end{document}
