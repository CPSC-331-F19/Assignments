\documentclass{article}
\usepackage[utf8]{inputenc}
\usepackage{amsmath}

\title{CPSC331-A1}
\author{Guransh Mangat, 30061719 }
\date{September 2019}

\begin{document}

\maketitle

\paragraph{Question 1}

\begin{enumerate}
    \item Since $n$ is an integer input, this is an \textit{integer-valued} function.\\ \\ When the recursive function is applied in \textit{line 9}, the value of n is reduced by at least 1 ($(n-1), (n-3) \ and \ (n-4)$). \newline \newline $n$ is a non negative integer, which means $n \geq 0$. Thus it is only possible that $ f(n) = n \leq 0$ when n = 0. In that case, the test at \textit{line 1} passes, the execution continues at \textit{line 1} and the execution ends without having to call the recursive function repeatedly. \newline \newline Hence, the function $f(n) = n$ is a bound function. 
    
    \item This theorem will be proved by induction on $n$. The strong form of mathematical induction will be used, and base cases of $n = 0, 1, 2 ,3$ will each be considered in the basis.\\
    
    Basis: Suppose, first $n = 0$: \newline During the execution of algorithm smacG on input $n$, the test at line 1 succeeds and the execution continues at line 2. This causes the execution to terminate with the value of $M_0 = 1$ returned at output - as required for this case \\ 
    
    Suppose, next, $n = 1$: \\ During the execution of the algorithm smacG, the test at line 1 fails and the execution continues with test at line 3. \\ This test succeeds, so the execution continues at line 4. This causes the execution of algorithm to terminate with value of $M_1 = 0$ returned as output- as required by this case \\ 
    
    Suppose, $n = 2$: \\ During the execution of the algorithm smacG, the test at line 1 and line 3 fails, and the execution continues with test at line 5. \\ This test succeeds, so the execution continues with line 6. This causes the execution of the algorithm to terminate with the value of $M_2 = 5$ returned as output - as require by this case. \\ 
    
    Suppose, n = 3: \\ During the execution of the algorithm smacG, the test at line 1, line 3 and line 5 fails, and the execution continues with the test at line 7. \\ This test succeeds, so the execution continues with line 8. This causes the execution of the algorithm to terminate with the value of $M_3 = 8$ returned as output - as required by this case. \\ 
    
    \textit{Inductive Step: } Let $k$ be an integer such that $k \geq 3$. \\
    
    \textbf{Inductive Hypothesis: } Suppose $n$ is a non-negative integer such that $ 0 \leq n \leq k$. If the algorithm smacG is executed given n as input then this execution of the algorithm eventually terminates, with the $n^{th}$ MacGonagall number, $M_n$ returned as output. \\
    
    \textbf{Inductive Claim: } If the algorithm smacG is executed given $n = k + 1$ as input then this execution of the algorithm eventually terminates, with the $k+1^{st}$ MacGonagall number $M_{k+1} = M_n$ returned as output. \\
    
    Suppose the algorithm smacG is executed with $n= k + 1$ given as input. Since $k$ is an integer such that $k \geq 3$, $n$ is an integer such that $n \geq 4$.
    
    Since $n \geq 4$, the test at line 1, line 3, line 5 and line 7 fails and the algorithm continues at line 8. 
    
    Line 9 includes a recursive execution of this algorithm with the input $2 \times (n-1)$. Since $n = k + 1 \geq 4, 0 \leq n-1 = k \leq k$, and it follows by the \textbf{inductive hypothesis} that this recursive execution of the algorithm eventually ends with $M_{n-1} = M_k$ returned as output
    
    Line 9 also includes a recursive execution of this algorithm with the input $-2 \times (n-3)$. Since $n = k+1 \geq 4, 0 \leq n-3 = k-2 \leq k$, and it follows the \textbf{inductive hypothesis} that this recursive execution of the algorithm eventually ends with $M_{n-3} = M_{k-2}$ returned as output.
    
    Line 9 also includes a recursive execution of algorithm with the input $n-4$. Since $n = k+1 \geq 4, 0 \leq n-4 = k-3 \leq 4$, and it follows by the \textbf{inductive hypothesis} that this recursive function of the algorithm eventually ends with $M_{n-4} = M_{k-3}$ returned as output.
    
    Once the execution of algorithm ends, since, $k+1 \geq 4$, it follows the definition of $M_{k+1}$, that the value returned as output is: \\ \\
    $2 \times M_{n-1} - 2 \times M_{n-3} + M_{n-4} = 2 \times M_{k} -2 \times M_{k-2} + M_{k-3} = M_{k+1}$ \\ 
    
    as required to establish the \textbf{inductive claim.} \\ \\
    This results now follows by induction on n.
    
    \item 
    
    \begin{verbatim}
package cpsc331.assignment1;

public class Main {

    public static void main(String[] args) {
        try {
            int n = Integer.parseInt(args[0]);
            if (n < 0) {
                System.out.println("Fiddlesticks! The integer input cannot be negative.");
            } else {
                System.out.println(smacG(n));
            }
        } catch (NumberFormatException e) {
            System.out.println("Fiddlesticks! One integer input is required.");
        }
    }


    private static int smacG(int n) throws IllegalArgumentException {

        if(n < 0) {
            throw new IllegalArgumentException();
        }

        switch (n) {
            case 0:
                return 1;
            case 1:
                return 0;
            case 2:
                return 5;
            case 3:
                return 8;
            default:
                return 2 * smacG(n - 1) - 2 * smacG(n - 3) + smacG(n - 4);
        }


    }
}
\end{verbatim}

\pagebreak

    \item
    
    In order to determine the number of steps included in the execution of $T_{smacG}(n)$, we will define the running time of each step in the algorithm to be \textit{one}.\\

    Using the uniform cost criterion to define $T_{smacG}(n)$: \\
    
    The algorithm executes 2 steps (at lines 1 and 2) if it is executed when $n = 0$. \\
    The algorithm executes 3 steps (at lines 1, 3, 4) if it is executed when $n = 1$. \\
    The algorithm executes 4 steps (at lines 1, 3, 5, 6) if it is executed when $n = 2$. \\
    The algorithm executes 5 steps (at lines 1, 3, 5, 7, 8) if it is executed when $n = 3$. \\
    
    
    \[ T_{smacG}(n) = \begin{cases} 
          2 & if n = 0, \\
          3 & if n = 1, \\
          4 & if n = 2, \\
          5 & if n = 3, \\
          T_{smacG}(n-1) + T_{smacG}(n-3) + T_{smacG}(n-4) + 5 & if n \geq 4.
       \end{cases}
    \]
    
    We can use this recurrence to show the following:\\
    
    $T_{smacG}(0)=2$: By definition of $T_{smacG}(n)$.\\ 
    $T_{smacG}(1)=3$: By definition of $T_{smacG}(n)$.\\ 
    $T_{smacG}(2)=4$: By definition of $T_{smacG}(n)$.\\ 
    $T_{smacG}(3)=5$: By definition of $T_{smacG}(n)$.\\ 
    
    $T_{smacG}(4)\\
    = T_{smacG}(4-1) + T_{smacG}(4-3) + T_{smacG}(4-4) + 5$\\
    $= T_{smacG}(3) + T_{smacG}(1) + T_{smacG}(0) + 5$ \\
    $= 5 + 3 + 2 + 5$ \\
    $= 15$. \\
    
    $T_{smacG}(5)\\
    = T_{smacG}(5-1) + T_{smacG}(5-3) + T_{smacG}(5-4) + 5$\\
    $= T_{smacG}(4) + T_{smacG}(2) + T_{smacG}(1) + 5$ \\
    $= 15 + 4 + 3 + 5$ \\
    $= 27$.
    
    \item
    
    \item 
    \begin{enumerate}
        \item \textit{n} is an integer input such that $n \geq 1$
        \item \textit{i} is an integer variable such that $1 \leq i \leq n$
        \item 
    \end{enumerate}
    
\end{enumerate}
    

\end{document}
